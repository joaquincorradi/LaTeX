\documentclass{report}
\usepackage[utf8]{inputenc}
\usepackage[nil, spanish]{babel}
\usepackage{amsmath, amsthm, amssymb, graphicx, thmtools, hyperref}
\renewcommand*\listtheoremname{Contenido}
\newtheorem*{theorem}{Teorema}
\newtheorem*{definition}{Definición}

\title{Teoremas y demostraciones}
\date{}
\author{Análisis matemático I}

\begin{document}
  \maketitle
  \listoftheorems
  \chapter*{Límites:} 
    \begin{definition}[límite formal]
      \begin{equation*}
        \lim_{x\to c}f(x)=L\iff\forall\epsilon>0 \ \exists\ \delta>0\ :0<|x-c|<\delta\Longrightarrow
        |f(x)-L|<\epsilon 
      \end{equation*}
    \end{definition}
    \begin{definition}[límite intuitivo]
      Decir que $\lim_{x\to c}f(x)=L$ significa  que cuando $x$ está cerca pero diferente de $c$,
      entonces $f(x)$ está cerca de $L$.
    \end{definition}
    \begin{theorem}[unicidad del límite]
      Si el límite de una función existe, entonces es único.
    \end{theorem}
    \begin{theorem}[del emparedado]
      Sean $f$, $g$ y $h$ funciones que satisfacen $f(x)\leqslant g(x)\leqslant h(x)\forall x$
      cercano a $c$, excepto posiblemente $c$. Si $\lim_{x\to c}f(x)=\lim_{x\to c}h(x)=L$, entonces
      $\lim_{x\to c}g(x)=L$.
    \end{theorem}
    \begin{proof}
      Sea $\epsilon>0$. Elegimos $\delta_1$ tal que
      \begin{equation*}
        0<|x-c|<\delta_1\Longrightarrow L-\epsilon<f(x)<L+\epsilon
      \end{equation*}
      y $\delta_2$ tal que
      \begin{equation*}
        0<|x-c|<\delta_2\Longrightarrow L-\epsilon<h(x)<L+\epsilon
      \end{equation*}
      Elegimos $\delta_3$ de modo que
      \begin{equation*}
        0<|x-c|<\delta_3\Longrightarrow f(x)\leqslant g(x)\leqslant h(x)
      \end{equation*}
      Sea $\delta=$mín\{$\delta_1,\delta_2,\delta_3$\}. Entonces
      \begin{equation*}
        0<|x-c|<\delta\Longrightarrow L-\epsilon<f(x)\leqslant g(x)\leqslant h(x)<L+\epsilon
      \end{equation*}
      Concluímos que $\lim_{x\to c}g(x)=L$
    \end{proof}

  \chapter*{Continuidad:}
    \begin{definition}[continuidad en un punto]
      Sea $f$ definida en un intervalo abierto que contiene a $c$. Decimos que $f$ es 
      continua en $c$ si
      \begin{equation*}
        \lim_{x\to c}f(x)=f(c)
      \end{equation*}
    \end{definition}
    \begin{theorem}[Bolzano]
      Sea $f$ una función continua y definida en $[a,b]$. Si se cumple que $f(a)<0<f(b)$ o 
      $f(b)<0<f(a)$, entonces existe un punto $c\in(a,b)$ tal que $f(c)=0$.
    \end{theorem}
    \begin{proof}
      Sea $f$ una función continua y definida en $[a,b]$ y $f(a)<0<f(b)$. Sea $C_+$ un
      conjunto tal que
      \begin{equation*}
        C_+=\{x\in[a,b]/f(x)\geqslant0\}
      \end{equation*}
      Sea $c\in[a,b]$ el supremo del conjuto $C_+$, entonces $\exists[c-\delta,c+\delta]=
      signo\ de\ f(c)$ (por teorema de la conservación del signo). Si suponemos que $f(c)<0$
      $c$ deja de ser una mínima cota superior. Si suponemos que $f(c)>0$ $c$ nuevamente deja
      de ser mínima cota superior. Entonces la única opción posible es que $f(c)=0$.
    \end{proof}
    \begin{theorem}[valor intermedio]
      Sea $f$ una función continua y definida en $[a,b]$ y $k\in(a,b)$ tal que $f(a)<k<f(b)$,
      entonces existe $c\in(a,b)$ tal que $f(c)=k$.
    \end{theorem}
    \begin{proof}
      Sea $f$ una función continua en $[a,b]$ y $k\in(a,b)$ tal que $f(a)<k<f(b)$. Sea
      $g(x)=f(x)-k$ entonces
      \begin{align*}
        g(a)&=f(a)-k\Longrightarrow g(a)<0\\
        g(b)&=f(b)-k\Longrightarrow g(b)>0
      \end{align*}
      Es decir que $g(a)<0<g(b)$ y por teorema de Bolzano existe un punto 
      $c\in(a,b)$ tal que $g(c)=0$, entonces
      \begin{align*}
        g(c)&=0\\
        g(c)&=f(c)-k\\
        0&=f(c)-k\\
        f(c)&=k
      \end{align*}
    \end{proof}
    \begin{theorem}[Weierstrass | máximos y mínimos]
      Sea $f$ una función continua y definida en $[a,b]$ entonces $\exists\ x_1,x_2\in[a,b]$ 
      tal que $f(x_1)\leqslant f(x)\leqslant f(x_2)\forall x\in[a,b]$.
    \end{theorem}

  \chapter*{Derivada:}
    \begin{definition}[derivada]
      La derivada de un función $f$ es otra función $f'$ cuyo valor en cualquier
      $x$ es
      \begin{equation*}
        f'(x)=\lim_{h\to 0}\frac{f(x+h)-f(x)}{h}=\frac{dy}{dx}
      \end{equation*}
    \end{definition}
    \begin{theorem}[diferenciabilidad implica continuidad]
      Si $f'(c)$ existe, entonces $f$ es continua en $c$.
    \end{theorem}
    \begin{proof}
      Sea $f$ una función tal que
      \begin{align*}
        f(x)&=f(x)\\
        f(x)&=f(c)+\frac{f(x)-f(c)}{x-c}\cdot(x-c)
      \end{align*}
      Por lo tanto,
      \begin{align*}
        \lim_{x\to c}f(x)&=\lim_{x\to c}\left[f(c)+\frac{f(x)-f(c)}{x-c}\cdot(x-c)\right]\\
                         &=\lim_{x\to c}f(c)+\lim_{x\to c}\frac{f(x)+f(c)}{x-c}\cdot
                         \lim_{x\to c}x-c\\
                         &=f(c)+f'(c)\cdot0\\
        \lim_{x\to c}f(x)&=f(c)
      \end{align*}
    \end{proof}
    \begin{theorem}[Fermat]
      Sea $f$ una función definida en $(a,b)$, si alcanza un máximo o mínimo local en $c$, y 
      si $f'(c)$ existe en el punto $c$, entonces $f'(c)=0$.
    \end{theorem}
    \begin{proof}
      Sea $f$ una función definida en $(a,b)$ y $c$ un máximo local.
      Supongamos que $\exists f'(c)$, entonces
      \begin{equation*}
        f'(c)=\lim_{h\to 0^+}\frac{f(c+h)-f(c)}{h}=\lim_{h\to 0^-}\frac{f(c+h)-f(c)}{h}
      \end{equation*}
      siendo
      \begin{equation*}
        f'(c)=\frac{f(c+h)-f(c)}{h}\leqslant0\text{ para }\lim_{h\to0^+}f(c)
      \end{equation*}
      y
      \begin{equation*}
        f'(c)=\frac{f(c+h)-f(c)}{h}\geqslant0\text{ para }\lim_{h\to0^-}f(c)
      \end{equation*}
      entonces,
      \begin{equation*}
        f'(c)=0
      \end{equation*}
    \end{proof}
    \begin{theorem}[Rolle]
      Sea $f$ una función continua en $[a,b]$ y derivable en $(a,b)$. Si $f(a)=f(b)$,
      entonces existe un punto $c\in(a,b)$ tal que $f'(c)=0$.
    \end{theorem}
    \begin{proof}
    \end{proof}
    \begin{theorem}[Lagrange]
    \end{theorem}
    \begin{theorem}[valor medio]
    \end{theorem}
    \begin{theorem}[L'Hopital]
    \end{theorem}
\end{document}
