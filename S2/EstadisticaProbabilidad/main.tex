\documentclass{report}
\usepackage[utf8]{inputenc}
\usepackage[nil, spanish]{babel}
\usepackage{amsmath, amsthm, amssymb, graphicx, hyperref}

\title{Resumen de estadística y probabilidad}
\date{Tercer semestre}
\author{J. I. Corradi}

\begin{document}
  \maketitle
  \setcounter{tocdepth}{5}
  \tableofcontents
  \newpage

  \addcontentsline{toc}{chapter}{Estadística}
  \chapter*{Estadística}

  \addcontentsline{toc}{section}{Conceptos básicos}
    \section*{Conceptos básicos}

      \addcontentsline{toc}{subsection}{Población}
      \subsection*{Población}

      \addcontentsline{toc}{subsection}{Población}
      \subsection*{Muestra}

    \addcontentsline{toc}{section}{Variables estadísticas}
    \section*{Variables estadísticas}

      \addcontentsline{toc}{subsection}{Cualitativas o categóricas}
      \subsection*{Cualitativas o categóricas}
        Son el tipo de variables que expresan distintas cualidades, características 
        o atributo.

        \addcontentsline{toc}{subsubsection}{Nominal}
        \subsubsection*{Nominal}
          Es aquella variable cualitativa cuya categoría no sigue ningún orden, 
          se agrupa sin ninguna jerarquía entre sí. Ej.: estado civíl, grupo 
          sanguíneo, sexo.
        \addcontentsline{toc}{subsubsection}{Ordinal}
        \subsubsection*{Ordinal} 
          Son aquellas variables categóricas con orden secuencial o progresión 
          natural esperable o jerarquía. Ej.: nivel de educación.

      \addcontentsline{toc}{subsection}{Cuantitativas o numéricas}
      \subsection*{Cuantitativas o numéricas}
        Describen una característica respecto a un valor numérico o cantidad.

        \addcontentsline{toc}{subsubsection}{Continuas}
        \subsubsection*{Continuas}
          Son aquellas características que se miden dentro de un rango continuo 
          infinito de valores numéricos y se registran con números reales. Pueden 
          presentar cualquier valor dentro de cierto intervalo. Ej.: estatura, 
          peso, ingresos.
        \addcontentsline{toc}{subsubsection}{Discretas}
        \subsubsection*{Discretas}
          Este tipo de variables numéricas están asociadas a conteos o enumeraciones, 
          razón por la cual solo pueden registrarse con números enteros. Ej.: numéro
          de hermanos, edad.
    \addcontentsline{toc}{section}{Medidas de localización}
    \section*{Medidas de localización}
      Las medidas de localización están diseñadas para brindar al analista algunos 
      valores cuantitativos de la ubicación central o de otro tipo de los datos en una 
      muestra.
      
      \addcontentsline{toc}{subsection}{Media}
      \subsection*{Media}
        La media es simplemente un promedio numérico.

        \begin{equation*}
          \overline{x}=\sum_{i = 1}^n\frac{x_i}{n}=\frac{x_1+x_2+\dotsm+x_i}{n}
        \end{equation*}

        \addcontentsline{toc}{subsection}{Mediana}
        \subsection*{Mediana}
          El propósito de la mediana de la muestra es reflejar la tendencia central 
          de la muestra de manera que no sea influida por los valores extremos. Los
          valores deben estar ordenados.\\
       
        \addcontentsline{toc}{subsection}{Cuartil}
        \subsection*{Cuartil}
          Los cuartiles son tres valores que dividen una muestra en cuatro partes
          iguales.
          
          \addcontentsline{toc}{subsubsection}{Primer cuartil}
          \subsubsection*{Primer cuartil}
            
            \begin{equation*}
              Q_1=\frac{n+1}{4}
            \end{equation*}

          \addcontentsline{toc}{subsubsection}{Tercer cuartil}
          \subsubsection*{Tercer cuartil}

        \addcontentsline{toc}{subsection}{Percentil}
        \subsection*{Percentil}
          El percentil es una medida que indica el valor de la variable por debajo 
          del cual se encuentra un porcentaje dado de observaciones en un grupo.

          \addcontentsline{toc}{subsubsection}{Segundo cuartil}
          \subsubsection*{Percentil 5}

          \addcontentsline{toc}{subsubsection}{Segundo cuartil}
          \subsubsection*{Percentil 95}

    \addcontentsline{toc}{section}{Medidas de variabilidad}
    \section*{Medidas de variabilidad}
\end{document}
