\documentclass{report}
\usepackage[utf8]{inputenc}
\usepackage[nil, spanish]{babel}
\usepackage{amsmath, amsthm, amssymb, graphicx, hyperref}

\title{Resumen de estadística y probabilidad}
\date{Tercer semestre}
\author{J. I. Corradi}

\begin{document}
  \maketitle
  \setcounter{tocdepth}{5}
  \tableofcontents
  \newpage

  \addcontentsline{toc}{chapter}{Estadística}
  \chapter*{Estadística}

  \addcontentsline{toc}{section}{Conceptos básicos}
    \section*{Conceptos básicos}

      \addcontentsline{toc}{subsection}{Población}
      \subsection*{Población}
        Una población estadística $N$ es el total de individuos o conjunto de 
        ellos que presentan o podrían presentar el rasgo característico que se 
        desea estudiar.

      \addcontentsline{toc}{subsection}{Población}
      \subsection*{Muestra}
      Una muestra $n$ es un subconjunto de casos o individuos de una población. 
      Interesa que una muestra sea representativa, y para ello debe escogerse 
      una técnica de muestra adecuada que produzca una muestra aleatoria adecuada.

      \addcontentsline{toc}{subsection}{Variable}
      \subsection*{Variable}
       Una variable es cualquier característica cuyo valor puede cambiar de un 
       objeto a otro en la población
         
    \addcontentsline{toc}{section}{Variables estadísticas}
    \section*{Variables estadísticas}

      \addcontentsline{toc}{subsection}{Cualitativas o categóricas}
      \subsection*{Cualitativas o categóricas}
        Son el tipo de variables que expresan distintas cualidades, características 
        o atributo.

        \addcontentsline{toc}{subsubsection}{Nominal}
        \subsubsection*{Nominal}
          Es aquella variable cualitativa cuya categoría no sigue ningún orden, 
          se agrupa sin ninguna jerarquía entre sí. Ej.: estado civíl, grupo 
          sanguíneo, sexo.
        \addcontentsline{toc}{subsubsection}{Ordinal}
        \subsubsection*{Ordinal} 
          Son aquellas variables categóricas con orden secuencial o progresión 
          natural esperable o jerarquía. Ej.: nivel de educación.

      \addcontentsline{toc}{subsection}{Cuantitativas o numéricas}
      \subsection*{Cuantitativas o numéricas}
        Describen una característica respecto a un valor numérico o cantidad.

        \addcontentsline{toc}{subsubsection}{Continuas}
        \subsubsection*{Continuas}
          Son aquellas características que se miden dentro de un rango continuo 
          infinito de valores numéricos y se registran con números reales. Pueden 
          presentar cualquier valor dentro de cierto intervalo. Ej.: estatura, 
          peso, ingresos.
        \addcontentsline{toc}{subsubsection}{Discretas}
        \subsubsection*{Discretas}
          Este tipo de variables numéricas están asociadas a conteos o enumeraciones, 
          razón por la cual solo pueden registrarse con números enteros. Ej.: numéro
          de hermanos, edad.
    \addcontentsline{toc}{section}{Medidas de localización o posición}
    \section*{Medidas de localización o posición}
      Las medidas de localización están diseñadas para brindar al analista algunos 
      valores cuantitativos de la ubicación central o de otro tipo de los datos en una 
      muestra.
      
      \addcontentsline{toc}{subsection}{Media}
      \subsection*{Media}
        La media es simplemente un promedio numérico.

        \begin{equation*}
          \overline{x}=\sum_{i = 1}^n\frac{x_i}{n}=\frac{x_1+x_2+\dotsm+x_i}{n}
        \end{equation*}

        \addcontentsline{toc}{subsection}{Mediana}
        \subsection*{Mediana}
          El propósito de la mediana de la muestra es reflejar la tendencia central 
          de la muestra de manera que no sea influida por los valores extremos. Los
          valores deben estar ordenados.\\
          \indent Para $n$ impar:
          \begin{equation*}
            M_e=\frac{2(n+1)}{4}
          \end{equation*}

          Para $n$ par:

          \begin{equation*}
            M_e=\frac{2n}{4}
          \end{equation*}
       
        \addcontentsline{toc}{subsection}{Cuartil}
        \subsection*{Cuartil}
          Los cuartiles son tres valores que dividen una muestra en cuatro partes
          iguales. Los valores deben estar ordenados.
          
          \addcontentsline{toc}{subsubsection}{Primer cuartil}
          \subsubsection*{Primer cuartil}
          
            \indent Para $n$ impar:
            \begin{equation*}
              Q_1=\frac{n+1}{4}
            \end{equation*}

            Para $n$ par:

            \begin{equation*}
              Q_1=\frac{n}{4}
            \end{equation*}

          \addcontentsline{toc}{subsubsection}{Tercer cuartil}
          \subsubsection*{Tercer cuartil}
            
            \indent Para $n$ impar:
            \begin{equation*}
              Q_3=\frac{3(n+1)}{4}
            \end{equation*}

            Para $n$ par:

            \begin{equation*}
              Q_3=\frac{3n}{4}
            \end{equation*} 

        \addcontentsline{toc}{subsection}{Percentil}
        \subsection*{Percentil}
          El percentil es una medida que indica el valor de la variable por debajo 
          del cual se encuentra un porcentaje dado de observaciones en un grupo.
          Los valores deben estar ordenados.

          \addcontentsline{toc}{subsubsection}{Segundo cuartil}
          \subsubsection*{Percentil 5}

            \indent Para $n$ impar:
            \begin{equation*}
              P_5=\frac{5(n+1)}{100}
            \end{equation*}

            Para $n$ par:

            \begin{equation*}
              P_5=\frac{5n}{100}
            \end{equation*}

          \addcontentsline{toc}{subsubsection}{Segundo cuartil}
          \subsubsection*{Percentil 95}

            \indent Para $n$ impar:
            \begin{equation*}
              P_95=\frac{95(n+1)}{100}
            \end{equation*}

            Para $n$ par:

            \begin{equation*}
              P_95=\frac{95n}{100}
            \end{equation*}

      \addcontentsline{toc}{subsection}{Moda}
      \subsection*{Moda}
        La moda es el valor más repetido del conjunto de datos, es decir, el valor 
        cuya frecuencia relativa es mayor. En un conjunto puede haber más de una 
        moda (bimodal).

    \addcontentsline{toc}{section}{Medidas de variabilidad o disperción}
    \section*{Medidas de variabilidad o disperción}
      
      \addcontentsline{toc}{subsection}{Varianza muestral}
      \subsection*{Varianza muestral}
        
        \begin{equation*}
          s^2=\sum_{i=1}^n\frac{(x_i-\overline{x})^2}{n-1}
        \end{equation*}

      \addcontentsline{toc}{subsection}{Desviación estándar}
      \subsection*{Desviación estándar}
        Es una medida del grado de disperción de de los datos con respecto al
        valor promedio.

        \begin{equation*}
          s=\sqrt{s^2}
        \end{equation*}

    \addcontentsline{toc}{section}{Frecuencia}
    \section*{Frecuencia}
      La frecuencia es una medida que sirve para comparar la aparición de un 
      elemento $x_i$ en un conjunto de elementos ($x_1, x_2, \dotsm, x_n$).\\
      \indent\textit{*acá falta colocar todos los tipos de frecuencias.*}

  \addcontentsline{toc}{chapter}{Probabilidad}
  \chapter*{Probabilidad}
    
    \addcontentsline{toc}{section}{Espacio muestral $\Omega$}
    \section*{Espacio muestral ($\Omega$)}
      Al conjunto de todos los resultados posibles de un experimento estadístico 
      se le llama espacio muestral $\Omega$ y se representa con el símbolo $S$.\\
      \indent Ej. 1: lanzar un dado.
      \begin{equation*}
        \Omega_1=\{1, 2, 3, 4, 5, 6\}
      \end{equation*}
      \indent Ej. 2: par o impar.
      \begin{equation*}
        \Omega_2=\{par, impar\}
      \end{equation*}
    \addcontentsline{toc}{section}{Evento}
    \section*{Evento}
    Un evento es el conjunto de uno o más resultados contenidos en el espacio 
    muestral $\Omega$. Un evento es simple si consiste en exactamente un resultado y 
    compuesto si consiste en más de un resultado.

    \addcontentsline{toc}{section}{Teoría de conjuntos}
    \section*{Teoría de conjuntos}

      \addcontentsline{toc}{subsection}{Complemento}
      \subsection*{Complemento}
      El complemento de un evento $A$, denotado por $A'$, es el conjunto de todos los 
      resultados en $\Omega$ que no están contenidos en $A$. 

      \addcontentsline{toc}{subsection}{Unión}
      \subsection*{Unión}
      La unión de dos eventos $A$ y $B$, denotados por $A\cup B$ y leídos \textit{"$A$ o $B$"}, 
      es el evento que consiste en todos los resultados que están en $A$ o en $B$ 
      o en ambos eventos, es decir, todos  los resultados en por lo menos uno de 
      los eventos. El evento $A\cup B$ ocurre si por lo menos ocurre $A$, $B$ 
      o ambos.

      \addcontentsline{toc}{subsection}{Intersección}
      \subsection*{Intersección}
      La intersección de dos eventos $A$ y $B$, denotada por $A\cap B$ y leída 
      \textit{"$A$ y $B$"}, es el evento que consiste en todos los resultados 
      que están tanto en $A$ como en $B$. El evento $A\cap B$ ocurre si $A$ y
      $B$ ocurren a la vez.

      \addcontentsline{toc}{subsection}{Evento nulo}
      \subsection*{Evento nulo}
      Se denomina $\varnothing$ a un evento sin resultados. Cuando $A\cap B=\varnothing$, se dice que 
      $A$ y $B$ son eventos mutuamente excluyentes o disjuntos.

    \addcontentsline{toc}{section}{Probabilidad}
    \section*{Probabilidad}
    
    \begin{equation*}
      P(A)=\frac{n_i}{N}
    \end{equation*}

\end{document}
