\documentclass[10pt, a4paper]{article}
\usepackage[utf8]{inputenc}
\usepackage{graphicx}
\usepackage[nil, spanish]{babel}

\begin{document}
  \begin{titlepage}
    \begin{center}
        \vspace*{1cm}
            
        \Huge
        \textbf{Historicidad de la existencia.}
            
        \vspace{0.5cm}
        \LARGE
        Trabajo práctico colaborativo.
            
        \vspace{1.5cm}
            
        \textbf{CORRADI, Joaquín I. - 2225513}\\
        \textbf{MERZ DURALDE, Conrado - 2313699}\\
        \textbf{FIDELIBUS, Nicolás - 2215795}
        \textbf{GARCÍA, Felipe - 2206422}\\
        \textbf{DIONISI MIRÓ, Renzo - 2208014}
            
        \vfill
            
        \vspace{0.8cm}
            
        \includegraphics[width=0.2\textwidth]{logoucc.png}
            
        \Large
        Antropología\\
        Universidad Católica de Córdoba\\
        9 de mayo de 2023
            
    \end{center}
\end{titlepage}


  \section*{Consignas del trabajo}

    \begin{enumerate}

      \item \begin{enumerate}

        \item Luego de leer el texto \textit{"Historicidad"}, especificamente lo
          referido a las tres temporalidades, ejemplificar cada una de las temporalidades
          con letras de canciones.

        \item Justificar.

      \end{enumerate}

      \item \begin{enumerate}

        \item ¿Cuál es la crítica de Gevaert al concepto de \textit{"determinismo"}?

        \item El autor describe los factores que justifican la ilusión determinista; 
          señalar un ejemplo, presente en la vida cotidiana contemporánea, para cada 
          uno de estos factores y justificar las relaciones.

      \end{enumerate}

      \item Completar un cuadro, teniendo en cuenta lo planteado por 
        el texto \textit{"Historicidad"} respecto al pasado, presente y futuro.

  \end{enumerate}

  \vspace{1.5cm}

  \section*{Resolución de consignas}
    
    \begin{enumerate}

      \item \begin{enumerate}
        \item La temporalidad restaurador se puede relacionar con la canción, la
          temporalidad constructor se puede relacionar con la canción, y la
          temporaliad utópico se puede relacionar con la canción
        \item b
      \end{enumerate} 

      \item \begin{enumerate}

        \item El autor critica la idea de \textit{"determinismo"} porque cree que 
          puede limitar a las personas. Él argumenta que pensar que todo en la 
          vida está predeterminado y que no tenemos control sobre nuestro destino 
          puede hacernos sentir impotentes y resignados. \\

          Asimismo, señala que el determinismo puede tener un impacto negativo en 
          la creatividad y la innovación, al no dejar espacio para la libre elección 
          y la imaginación. También destaca que puede tener implicaciones éticas 
          y morales problemáticas, al librarse de responsabilidad personal en 
          las decisiones y acciones.\\

          Desde la perspectiva del autor, el determinismo es una visión simplista 
          de la realidad, que no tiene en cuenta su complejidad. En cambio, apoya 
          una visión más realista, que reconozca la importancia de la libre elección, 
          la responsabilidad personal y la complejidad de la vida humana.

        \item Los factores que determinan la ilusión determinista son los siguientes:\\ 
          lusión retrospectiva. A menudo pensamos que los eventos pasados eran 
          inevitables porque ya sucedieron y no se pueden cambiar. Por ejemplo, 
          la caída del Muro de Berlín en 1989 se debe a tensiones políticas y 
          económicas que se habían estado gestando en Europa del Este durante 
          mucho tiempo. Esta idea puede surgir porque no podemos cambiar el 
          pasado.\\

          Impotencia frente a los acontecimientos. A veces las personas sienten 
          que no pueden cambiar eventos importantes de la historia, como el cambio 
          climático. Aunque intenten hacer algo al respecto, parece que estos eventos 
          van a suceder de todas maneras, como si estuvieran destinados a ocurrir.\\ 

          Deseo de ofrecer una base ideológica a un compromiso histórico. Hay personas 
          que creen en el determinismo histórico para justificar su lucha o 
          compromiso histórico. El marxismo es un ejemplo de esta idea, ya que 
          sostiene que la historia se desarrolla a través de las luchas de clases 
          y hacia la victoria del proletariado. Para el comunismo, esta idea es 
          importante porque la libertad individual podría poner en peligro la 
          teoría marxista y su visión de la sociedad perfecta.\\
          
          Creencia en el mandamiento divino. Algunas personas creen que Dios 
          ha planeado todo lo que sucede en la historia y que los eventos ocurren 
          de una manera específica. Esto puede hacer que piensen que los humanos 
          no tienen mucho control sobre lo que sucede y que los desastres naturales 
          y los conflictos armados son parte del plan de Dios y no pueden ser cambiados. 
      \end{enumerate}
        
    \end{enumerate}
    
\end{document}
