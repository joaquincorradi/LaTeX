\documentclass[12pt, a4paper]{article}
\usepackage[utf8]{inputenc}
\usepackage{graphicx}
\usepackage[nil, spanish]{babel}

\newenvironment{localsize}[1]
{%
  \let\orignewcommand\newcommand
  \let\newcommand\renewcommand
  \makeatletter
  \input{bk#1.clo}%
  \makeatother
  \let\newcommand\orignewcommand
}

\begin{document}
    \begin{titlepage}
    \begin{center}
        \vspace*{1cm}
            
        \Huge
        \textbf{Historicidad de la existencia.}
            
        \vspace{0.5cm}
        \LARGE
        Trabajo práctico colaborativo.
            
        \vspace{1.5cm}
            
        \textbf{CORRADI, Joaquín I. - 2225513}\\
        \textbf{MERZ DURALDE, Conrado - 2313699}\\
        \textbf{FIDELIBUS, Nicolás - 2215795}
        \textbf{GARCÍA, Felipe - 2206422}\\
        \textbf{DIONISI MIRÓ, Renzo - 2208014}
            
        \vfill
            
        \vspace{0.8cm}
            
        \includegraphics[width=0.2\textwidth]{logoucc.png}
            
        \Large
        Antropología\\
        Universidad Católica de Córdoba\\
        9 de mayo de 2023
            
    \end{center}
\end{titlepage}


  \section*{Consignas del trabajo}

    \begin{enumerate}

      \item Antes de realizar esta actividad, ¿Conocían el presente Protocolo? Si la respuesta es afirmativa, señalar cómo accedieron a esta información. Si la respuesta es no: señalar al menos dos estrategias de visibilización de dicho protocolo en la Facultad de Ingeniería.

      \item Según sus opiniones: ¿cuáles son las fortalezas del presente protocolo y, ¿Cuáles son sus debilidades? Señalar de modo claro y específico.

      \item ¿Qué tipos de violencias descritas por el protocolo en su artículo 5 y 6 son las más comunes en sus contextos cotidianos? Justificar.

      \item ¿Creen que en su Facultad hay o hubo casos de violencia de género? Describirlos, cuidando el anonimato de los protagonistas. Tener muy en cuenta los artículos 5, 6 y el punto final “Transitorias”.

      \item ¿Creen que conocer y trabajar en este protocolo institucional tiene que ver con alguna temática que estamos estudiando en Antropología? En caso de respuesta afirmativa o negativa. Justificar.

  \end{enumerate}

  \vspace{1.5cm}

  \section*{Resolución de consignas}
    
    \begin{enumerate}

      \item Antes de llevar a cabo la actividad, ninguno de los miembros del grupo tenía conocimiento del protocolo. Consideramos que se podrían implementar diversas estrategias con el objetivo de difundir el protocolo entre las diferentes partes que conviven en la universidad.\\
        En primer lugar, proponemos realizar charlas informativas. Esta opción tiene la ventaja de proporcionar información detallada sobre el tema. Los distintos puntos del protocolo podrían ser explicados en profundidad por personas capacitadas en el tema, como los miembros de la Comisión. Además, se crearía un espacio abierto para que los participantes puedan plantear preguntas.\\
        En segundo lugar, proponemos llevar a cabo una campaña en redes sociales. Consideramos que esta opción es la más efectiva y viable debido a su amplio alcance, ya que llega tanto a los miembros de la universidad como a personas externas interesadas en el tema. Las redes sociales también ofrecen la ventaja de una difusión rápida. Además, se podría incluir contenido audiovisual relacionado con el protocolo, y las redes sociales facilitan el hecho de compartir dicho contenido, haciéndolo rápido y sencillo.\\

      \item El protocolo tiene como fortaleza su descripción detallada de los pasos a seguir para denunciar o reportar cualquier forma de violencia. Además, destaca el hecho de que asigna una comisión específica para tratar estos casos.\\
        Sin embargo, una debilidad del protocolo es que no aclara desde el principio que se aplica a todos los tipos de violencia, no solo a aquella dirigida hacia las mujeres. Esta información se menciona brevemente al final, lo que implica que se espera que el lector lea todo el documento para obtener dicha información. No se proporciona ningún indicio sobre los distintos tipos de violencia que no estén relacionados con cuestiones de género.
 
      \item La violencia que observamos más a diario en nuestra vida cotidiana es la del tipo simbólica. Hay muchos estereotipos de género arraigados en nuestro contexto como el de las carreras para hombres y mujeres, que ejemplificamos en el siguiente punto, y otros que se escuchan mucho como por ejemplo “las mujeres manejan mal”. También vemos la violencia simbólica en el lenguaje sexista, que incluso nosotros usamos. Esta la vemos principalmente en los insultos en donde se hace referencia a un género o el otro como algo negativo.

      \item En el contexto universitario, lamentablemente, hemos presenciado actos de violencia, destacando en particular el trato "especial" que algunos profesores brindan a las mujeres. Esto se justifica con la idea de que la ingeniería es una carrera dominada por hombres y, por lo tanto, se les debe dar un trato diferenciado.. Sin embargo, esta justificación muchas veces roza el acoso, generando evidente incomodidad en nuestras compañeras. En esta situación, de acuerdo con el artículo 5 del protocolo, nos encontramos frente a un caso de violencia simbólica.

      \item Creemos que el protocolo se relaciona en parte con los temas que vimos anteriormente en la materia, principalmente lo pudimos relacionar con la propia definición de antropología, ya que como vimos esta se ocupa de comprender la cultura y la sociedad.. Se puede decir que la violencia de género es un fenómeno que está arraigado en la sociedad y cultura, y se manifiesta de diferentes maneras en distintas sociedades, por lo tanto, es necesaria la presencia de la antropología en el tema de la violencia de género, ya que es necesaria para poder entender su origen y las diferentes formas en las que se puede abordar.

    \end{enumerate}
    
\end{document}
