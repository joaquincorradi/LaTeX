\documentclass[a4, 12pt]{report}
\usepackage[utf8]{inputenc}
\usepackage[nil, spanish]{babel}
\usepackage{amsmath, amsthm, amssymb, graphicx, hyperref}

\title{Listado de formulas de Física II - Segundo parcial}
\date{Tercer semestre}
\author{CORRADI, Joaquín I.}

\begin{document}
  \maketitle
  \setcounter{tocdepth}{5}
  \tableofcontents
  \newpage

  \addcontentsline{toc}{chapter}{Campos magnéticos}
  \chapter*{Campos magnéticos}

    \addcontentsline{toc}{section}{Fuerza magnética}
    \section*{Fuerza magnética}
      
      \begin{align*}
        \vec{F}_B&=q\vec{v}\times\vec{B}\\
        F_B&=qvB\sin{\theta}
      \end{align*}

      Siendo $\vec{B}$ el vector de campo magnético, $q$ una partícula cargada
      y $v$ la velocidad de esta.\\
      \indent$B$ se mide en Tesla.

    \addcontentsline{toc}{section}{Interacción de $B$ con una corriente $I$}
    \section*{Interacción de $B$ con una corriente $I$}

    \begin{align*}
      d\vec{F}_B=Id\vec{s}\times\vec{B}\\
      F_B=IsB\sin{\theta}
    \end{align*}

    \addcontentsline{toc}{section}{Radio de la trayectoria circular de una particula en un campo}
    \section*{Radio de la trayectoria circular de una particula en un campo}

      \begin{equation*}
        r=\frac{mv}{qB}
      \end{equation*}

      \indent Donde $m$ es la masa de la partícula y $q$ su carga.

  \addcontentsline{toc}{chapter}{Fuentes de campos magnéticos}
  \chapter*{Fuente de campos magnéticos}

    \addcontentsline{toc}{section}{Conductor corto}
    \section*{Conductor corto}

    \begin{equation*}
      B_P=\frac{\mu_0I}{4\pi a}(\cos{\theta_1}-\cos{\theta_2})
    \end{equation*}

    \indent Siendo $a$ la distancia del conductor al punto.

    \addcontentsline{toc}{section}{Conductor largo}
    \section*{Conductor largo}

      \begin{equation*}
        B_P=\frac{\mu_0I}{2\pi a}
      \end{equation*}

      \indent Siendo $a$ la distancia del conductor al campo.

    \addcontentsline{toc}{section}{Espira}
    \section*{Espira}

      \begin{equation*}
        B_O=\frac{\mu_0I}{4\pi R}\theta
      \end{equation*}

    \addcontentsline{toc}{section}{Bobina toroidal}
    \section*{Bobina toroidal}

      \begin{equation*}
        B=\frac{\mu_0NI}{2\pi r}
      \end{equation*}

      \indent Siendo $r$ el radio varible.

    \addcontentsline{toc}{section}{Solenoide}
    \section*{Solenoide}

      \begin{equation*}
        B_P=\frac{\mu_0NI}{2l}(\sen{\theta_2}-\sen{\theta_1})
      \end{equation*}

      \indent En caso de ser un solenoide largo la formula es la siguiente

      \begin{align*}
        B &= \frac{\mu_0NI}{l}\\
          &= \mu_0nI
      \end{align*}

      \indent Si el punto $P$ se encuentra en el extremo superior de un solenoide
      largo entonces $\theta_2 = 0$ y $\theta_1 = -90$.

      \begin{equation*}
        B_{extremo}=\frac{\mu_0nI}{2}
      \end{equation*}


    \addcontentsline{toc}{section}{Fuerza entre dos conductores paralelos}
    \section*{Fuerza entre dos conductores paralelos}

      \begin{equation*}
        \frac{F}{l}=\frac{\mu_0 I_1I_2}{2\pi a}
      \end{equation*}

      \indent Siendo $a$ la distancia entre los dos conductores y $l$ la longitud
      de los conductores.

    \addcontentsline{toc}{section}{Flujo magnético}
    \section*{Flujo magnético}

      \begin{align*}
        \Phi_m&=\int\vec{B}\cdot d\vec{A}\\
        &=\int BA\cos{\theta}
      \end{align*}

  \addcontentsline{toc}{chapter}{Ley de Faraday}
  \chapter*{Ley de Faraday}

    \addcontentsline{toc}{section}{Definición}
    \section*{Definición}

      \begin{equation*}
        \mathcal{E}=-\frac{d}{dt}\Phi_m
      \end{equation*}

      \indent Siedo $\mathcal{E}$ la F.E.M inducida en una espira.

    \addcontentsline{toc}{section}{Ley de Faraday para $N$ espiras}
    \section*{Ley de Faraday para $N$ espiras}

      \begin{equation*}
        \mathcal{E}=-N\frac{d}{dt}\Phi_m
      \end{equation*}

    \addcontentsline{toc}{section}{F.E.M. de movimiento}
    \section*{F.E.M. de movimiento}

    \begin{equation*}
      \mathcal{E}=-Blv
    \end{equation*}

    \addcontentsline{toc}{section}{Ley de Ohm}
    \section*{Ley de Ohm}
    
      \begin{equation*}
        I=\frac{\mathcal{E}}{R}
      \end{equation*}
      
      \indent Siendo $R$ la resistencia.

\end{document}
