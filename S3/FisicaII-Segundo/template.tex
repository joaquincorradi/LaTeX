\documentclass[a4, 12pt]{report}
\usepackage[utf8]{inputenc}
\usepackage[nil, spanish]{babel}
\usepackage{amsmath, amsthm, amssymb, graphicx, hyperref}

\title{Listado de formulas de Física II - Segundo parcial}
\date{Tercer semestre}
\author{CORRADI, Joaquín I.}

\begin{document}
  \maketitle
  \setcounter{tocdepth}{5}
  \tableofcontents
  \newpage

  \addcontentsline{toc}{chapter}{Campos magnéticos}
  \chapter*{Campos magnéticos}

    \addcontentsline{toc}{section}{Fuerza magnética}
    \section*{Fuerza magnética}
      
      \begin{align*}
        \vec{F}_B&=q\vec{v}\times\vec{B}\\
        |F_B|&=|q||v||B|\sin{\theta}
      \end{align*}

      Siendo $\vec{B}$ el vector de campo magnético, $q$ una partícula cargada
      y $v$ la velocidad de esta.\\
      \indent$B$ se mide en Tesla

      \begin{equation*}
        1T=1\frac{N}{C\frac{m}{s}}=1\frac{N}{Am}
      \end{equation*}

    \addcontentsline{toc}{section}{Interacción de $B$ con una corriente $I$}
    \section*{Interacción de $B$ con una corriente $I$}

    \begin{align*}
      d\vec{F}_B=Id\vec{S}\times\vec{B}\\
      F_B=ISB\cos{\theta}
    \end{align*}

  \addcontentsline{toc}{chapter}{Fuentes de campos magnéticos}
  \chapter*{Fuente de campos magnéticos}

    \addcontentsline{toc}{section}{Conductor corto}
    \section*{Conductor corto}

    \begin{equation*}
      B_P=\frac{\mu_0I}{4\pi a}(\cos{\theta_1}-\cos{\theta_2})
    \end{equation*}

    \addcontentsline{toc}{section}{Conductor largo}
    \section*{Conductor largo}

      \begin{equation*}
        B_P=\frac{\mu_0I}{2\pi a}
      \end{equation*}

    \addcontentsline{toc}{section}{Espira}
    \section*{Espira}

      \begin{equation*}
        B_O=\frac{\mu_0I}{4\pi R}\theta
      \end{equation*}

    \addcontentsline{toc}{section}{Bobina toroidal}
    \section*{Bobina toroidal}

      \begin{equation*}
        B=\frac{\mu_0NI}{2\pi r}
      \end{equation*}

    \addcontentsline{toc}{section}{Solenoide}
    \section*{Solenoide}

      \begin{equation*}
        B_P=\frac{\mu_0NI}{2l}(\sen{\theta_2}-\sen{\theta_1})
      \end{equation*}

    \addcontentsline{toc}{section}{Fuerza entre dos conductores paralelos}
    \section*{Fuerza entre dos conductores paralelos}

      \begin{equation*}
        \frac{F}{L}=\frac{\mu_0 I_1I_2}{2\pi a}
      \end{equation*}

    \addcontentsline{toc}{section}{Flujo magnético}
    \section*{Flujo magnético}

      \begin{equation*}
        \Phi_m=\int\vec{B}\cdot d\vec{A}
      \end{equation*}

  \addcontentsline{toc}{chapter}{Ley de Faraday}
  \chapter*{Ley de Faraday}

    \addcontentsline{toc}{section}{Definición}
    \section*{Definición}

      \begin{equation*}
        \mathcal{E}=-\frac{d\Phi_B}{dt}
      \end{equation*}

    \addcontentsline{toc}{section}{}
    \section*{}

\end{document}
