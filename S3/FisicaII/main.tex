\documentclass{report}
\usepackage[utf8]{inputenc}
\usepackage[nil, spanish]{babel}
\usepackage{amsmath, amsthm, amssymb, graphicx, hyperref}

\title{Listado de fórmulas y conceptos de física II}
\date{Tercer semestre}
\author{Joaquín I. Corradi}

\begin{document}
  \maketitle
  \setcounter{tocdepth}{5}
  \tableofcontents
  \newpage

  \addcontentsline{toc}{section}{Campo eléctrico}
  \section*{Campo eléctrico}

    \addcontentsline{toc}{subsection}{Ley de Coulomb}
    \subsection*{Ley de Coulomb}
      La ley de Coulomb establece que la fuerza entre dos cargas eléctricas es directamente 
      proporcional al producto de sus magnitudes y inversamente proporcional al cuadrado de 
      la distancia entre ellas. Esta ley describe cómo se atraen o se repelen las cargas eléctricas.
      \begin{equation*}
        F_{12}=k_e\frac{|q_1||q_2|}{r^2}[N]
      \end{equation*}

    \addcontentsline{toc}{subsection}{Constante de Coulomb}
    \subsection*{Constante de Coulomb}
      \begin{equation*}
        k_e=8,9876\times10^9[N]=\frac{1}{4\pi\varepsilon_0}
      \end{equation*}

    \addcontentsline{toc}{subsection}{Permitividad del vacío}
    \subsection*{Permitividad del vacío}
      \begin{equation*}
        \varepsilon_0=8,8542\times10^{-12}\left[\frac{C^2}{NM^2}\right]
      \end{equation*}

    \addcontentsline{toc}{subsection}{Definición de campo eléctrico}
    \subsection*{Definición de campo eléctrico}
      El campo eléctrico es una propiedad del espacio que rodea una carga eléctrica $q$ y puede ejercer una 
      fuerza eléctrica sobre otra carga de prueba $q_0$ cercana. Se describe mediante vectores que indican la 
      magnitud y dirección de la fuerza eléctrica experimentada por una carga de prueba.
      \begin{equation*}
        \vec{E}=\frac{\vec{F}_e}{q_0}\left[\frac{N}{C}\right]
      \end{equation*}

    \addcontentsline{toc}{subsection}{Campo eléctrico de una carga puntual $q$}
    \subsection*{Campo eléctrico de una carga puntual $q$}
      \begin{equation*}
        E=k\frac{q}{r^2}\hat{r}\left[\frac{N}{C}\right]
      \end{equation*}

    \addcontentsline{toc}{subsection}{Campo eléctrico de un grupo de cargas}
    \subsection*{Campo eléctrico de un grupo de cargas}
      \begin{equation*}
        E=k\sum_{i}\frac{q_i}{r_i^2}\hat{r}_i\left[\frac{N}{C}\right]
      \end{equation*}

    \addcontentsline{toc}{subsection}{Campo eléctrico de una distribución de carga continua}
    \subsection*{Campo eléctrico de una distribución de carga continua}
      \begin{equation*}
        E=k_e\int\frac{dq}{r^2}\hat{r}\left[\frac{N}{C}\right]
      \end{equation*}

    \addcontentsline{toc}{subsection}{Aceleración de una carga en un campo eléctrico}
    \subsection*{Aceleración de una carga en un campo eléctrico}
      \begin{equation*}
        a=\frac{qE}{m}
      \end{equation*}

    % \addcontentsline{toc}{subsection}{Fuerza sobre una partícula con carga colocada en un campo eléctrico}
    % \subsection*{Fuerza ejercida sobre una partícula con carga $q$ colocada en un campo eléctrico}
    %   \begin{equation*}
    %     \overrightarrow{F}_e=q_0\overrightarrow{E}[N]
    %   \end{equation*}
    %
    % \addcontentsline{toc}{subsection}{Fuerza sobre una carga de pruebas según la ley de Coulomb}
    % \subsection*{Fuerza ejercida por $q$ sobre una carga de pruebas según la ley de Coulomb}
    %   \begin{equation*}
    %     \overrightarrow{F}_e=k_e\frac{qq_0}{r^2}\hat{r}[N]
    %   \end{equation*}
    %   Dónde $\hat{r}$ es un vector unitario con dirección de $q$ hacia $q_0$.


  \addcontentsline{toc}{section}{Ley de Gauss}
  \section*{Ley de Gauss}

    \addcontentsline{toc}{subsection}{Definición de flujo eléctrico}
    \subsection*{Definición de flujo eléctrico}
      El flujo eléctrico es una medida de la cantidad de líneas de campo eléctrico que atraviesan 
      una superficie dada. Se puede visualizar como la cantidad de líneas que salen o entran en una superficie
      \begin{equation*}
        \Phi_E=\oint_s\vec{E}\cdot d\vec{A}=\left[\frac{NM^2}{C}\right]
      \end{equation*}

    \addcontentsline{toc}{subsection}{Flujo eléctrico para superficie sencilla}
    \subsection*{Flujo eléctrico para superficie sencilla}
      \begin{equation*}
        \Phi_E=\vec{E}\cdot\vec{A}\left[\frac{NM^2}{C}\right]=EA\cos{\theta}\left[\frac{NM^2}{C}\right]
      \end{equation*}
    Siendo $\vec{S}$ el vector perpendicular a la superficie y $\theta$ el ángulo conformado por los vectores
    $\vec{E}$ y $\vec{A}$.

    \addcontentsline{toc}{subsection}{Ley de Gauss}
    \subsection*{Ley de Gauss}
      \begin{align*}
        \Phi_E &=\frac{\sum q_{int}}{\varepsilon_0}\left[\frac{NM^2}{C}\right]\\
        \oint_s\vec{E}\cdot d\vec{A}&=\frac{\sum q_{int}}{\varepsilon_0}\left[\frac{NM^2}{C}\right]\\
        \Phi_E &=\oint_s\vec{E}\cdot d\vec{A}\left[\frac{NM^2}{C}\right]
      \end{align*}

  \addcontentsline{toc}{section}{Potencial eléctrico}
  \section*{Potencial eléctrico}
    
    \addcontentsline{toc}{subsection}{Definición de potencial eléctrico}
    \subsection*{Definición de potencial eléctrico}
      Trabajo que debe realizar un campo eléctrico para mover una carga positiva desde dicho punto hasta el punto de 
      referencia, dividido por unidad de carga de prueba.
      \begin{equation*}
        dw=\vec{F}\cdot\vec{ds}=q_0\vec{E}\cdot\vec{ds}[J]
      \end{equation*}

    \addcontentsline{toc}{subsection}{Diferencia de potencial entre $A$ y $B$}
    \subsection*{Diferencia de potencial entre $A$ y $B$}
      
      \begin{align*}
        V_B-V_A=\frac{U_B-U_A}{q_0}=-\int_A^B\vec{E}\cdot\vec{ds}\\
        \Delta V=\frac{\Delta U}{q_0}\left[\frac{J}{C}=V\right]=-\int\vec{E}\cdot\vec{ds}\left[\frac{N}{C}m=V\right]
      \end{align*}

    \addcontentsline{toc}{subsection}{Diferencia de potencial en un campo $E$ uniforme}
    \subsection*{Diferencia de potencial en un campo $E$ uniforme}
      
      \begin{equation*}
        \Delta V=-\vec{E}\cdot\vec{d}=-|E||d|\cos{\theta}
      \end{equation*}

    \addcontentsline{toc}{subsection}{Potencial eléctrico y energía potencial debido a una carga puntual}
    \subsection*{Potencial eléctrico y energía potencial debido a una carga puntual}

      \begin{align*}
        V=k\frac{q}{r}\\
        U=q_2v_1=k\frac{q_1q_2}{r_{12}}
      \end{align*}

    \addcontentsline{toc}{subsection}{Relación entre $E$ y $V$}
    \subsection*{Relación entre $E$ y $V$}

      \begin{align*}
        E_x=-\frac{\partial V}{\partial x}\\
        E_y=-\frac{\partial V}{\partial y}\\
        E_z=-\frac{\partial V}{\partial z}\\
      \end{align*}

    \addcontentsline{toc}{subsection}{Potencial debido a una distribución de carga}
    \subsection*{Potencial debido a una distribución de carga}

      \begin{equation*}
        V=k\int\frac{dq}{r}
      \end{equation*}

  \addcontentsline{toc}{section}{Capacitancia y capacitores}
  \section*{Capacitancia y capacitores}

    \addcontentsline{toc}{subsection}{Capacitancia}
    \subsection*{Capacitancia}

      \begin{equation*}
        C=\frac{Q}{V}\left[\frac{C}{V}=Faradio\right]
      \end{equation*}

    \addcontentsline{toc}{subsection}{Capacitancia de una esfera cargada aislada de radio $R$}
    \subsection*{Capacitancia de una esfera cargada aislada de radio $R$}

      \begin{equation*}
        C=4\pi\varepsilon_0R
      \end{equation*}

    \addcontentsline{toc}{subsection}{Capacitor de placas paralelas separadas por $d$ y área de placas $A$}
    \subsection*{Capacitor de placas paralelas separadas por $d$ y área de placas $A$}

      \begin{equation*}
        C=\frac{\varepsilon_0A}{d}
      \end{equation*}

    \addcontentsline{toc}{subsection}{Capacitor cilíndrico}
    \subsection*{Capacitor cilíndrico}

      \begin{equation*}
        C=\frac{l}{2kln(\frac{b}{a})}
      \end{equation*}

    \addcontentsline{toc}{subsection}{Capacitor esférico}
    \subsection*{Capacitor esférico}

      \begin{equation*}
        C=\frac{ab}{k(b-a)}
      \end{equation*}

    \addcontentsline{toc}{subsection}{Combinación de capacitores}
    \subsection*{Combinación de capacitores}
      
      En paralelo:\\
      \begin{equation*}
        C_{eq}=C_1+C_2
      \end{equation*}
      \indent En serie:\\
      \begin{equation*}
        \frac{1}{C_{eq}}=\frac{1}{C_1}+\frac{1}{C_2}
      \end{equation*}

    \addcontentsline{toc}{subsection}{Energía almacenada en un capacitor}
    \subsection*{Energía almacenada en un capacitor}

      \begin{equation*}
        U=\frac{1}{2}\frac{Q^2}{C}=\frac{1}{2}QV=\frac{1}{2}CV^2[J]
      \end{equation*}

  \addcontentsline{toc}{section}{Corriente y resistencia}
  \section*{Corriente y resistencia}

    \addcontentsline{toc}{subsection}{Corriente eléctrica}
    \subsection*{Corriente eléctrica}

      \begin{equation*}
        I=\frac{dq}{dt}\left[\frac{C}{seg}=A\right]
      \end{equation*}

    \addcontentsline{toc}{subsection}{Resistencia}
    \subsection*{Resistencia}

      \begin{equation*}
        R=\frac{V}{I}\left[\frac{V}{A}=\Omega\right]
      \end{equation*}

    \addcontentsline{toc}{subsection}{Resistencia de un conductor uniforme}
    \subsection*{Resistencia de un conductor uniforme}

      \begin{equation*}
        R=\rho\frac{l}{A}[\Omega]
      \end{equation*}
      \indent Siendo $\rho$ el coeficiente de resistividad, $l$ la longitud que recorrería
      la corriente eléctrica aplicada una diferencia de potencial y $A$ la sección
      perpendicular a la dirección de la corriente.

    \addcontentsline{toc}{subsection}{Variación de la resistividad con la temperatura}
    \subsection*{Variación de la resistividad con la temperatura}
      
      \begin{equation*}
        \rho=\rho_0[1+\alpha(\Delta T)]
      \end{equation*}
      \indent Siendo $\rho_0$ la resistividad a una temperatura de $20^{\circ}$ y $\alpha$
      el coeficiente de temperatura.
      \begin{equation*}
        R=R_0[1+\alpha(\Delta T)]
      \end{equation*}

    \addcontentsline{toc}{subsection}{Conductividad}
    \subsection*{Conductividad}

      \begin{equation*}
        \sigma=\frac{1}{\rho}[\Omega^{-1}m^{-1}]
      \end{equation*}

    \addcontentsline{toc}{subsection}{Densidad de corriente}
    \subsection*{Densidad de corriente}

      \begin{equation*}
        J=\frac{I}{A}\left[\frac{A}{m^2}\right]
      \end{equation*}

    \addcontentsline{toc}{subsection}{Ley de Ohm}
    \subsection*{Ley de Ohm}

      \begin{equation*}
        J=\sigma E
      \end{equation*}

    \addcontentsline{toc}{subsection}{Potencia}
    \subsection*{Potencia}

      \begin{equation*}
        P[W]=IV=I^2R=\frac{V^2}{R}
      \end{equation*}

  \addcontentsline{toc}{section}{Circuitos de corriente directa}
  \section*{Circuitos de corriente directa}

    \addcontentsline{toc}{subsection}{Voltaje entre bornes de una batería}
    \subsection*{Voltaje entre bornes de una batería}

      \begin{equation*}
        V=\mathcal{E}-Ir
      \end{equation*}

    \addcontentsline{toc}{subsection}{Combinación de resistencias}
    \subsection*{Combinación de resistencias}
      
      En paralelo:\\
      \begin{equation*}
        \frac{1}{R_{eq}}=\frac{1}{R_1}+\frac{1}{R_2}
      \end{equation*}
      \indent En serie:
      \begin{equation*}
        R_{eq}=R_1+R_2
      \end{equation*}

\end{document}
