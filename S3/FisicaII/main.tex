\documentclass{report}
\usepackage[utf8]{inputenc}
\usepackage[nil, spanish]{babel}
\usepackage{amsmath, amsthm, amssymb, graphicx, hyperref}

\title{Listado de fórmulas y conceptos de física II}
\date{}
\author{Joaquín I. Corradi}

\begin{document}
  \maketitle
  \setcounter{tocdepth}{5}
  \tableofcontents
  \newpage

  \addcontentsline{toc}{section}{Campo eléctrico}
  \section*{Campo eléctrico}

    \addcontentsline{toc}{subsection}{Ley de Coulomb}
    \subsection*{Ley de Coulomb}
      La ley de Coulomb establece que la fuerza entre dos cargas eléctricas es directamente 
      proporcional al producto de sus magnitudes y inversamente proporcional al cuadrado de 
      la distancia entre ellas. Esta ley describe cómo se atraen o se repelen las cargas eléctricas.
      \begin{equation*}
        F_{12}=k_e\frac{|q_1||q_2|}{r^2}[N]
      \end{equation*}

    \addcontentsline{toc}{subsection}{Constante de Coulomb}
    \subsection*{Constante de Coulomb}
      \begin{equation*}
        k_e=8,9876\times10^9[N]=\frac{1}{4\pi\epsilon_0}
      \end{equation*}

    \addcontentsline{toc}{subsection}{Permitividad del vacío}
    \subsection*{Permitividad del vacío}
      \begin{equation*}
        \epsilon_0=8,8542\times10^{-12}\left[\frac{C^2}{NM^2}\right]
      \end{equation*}

    \addcontentsline{toc}{subsection}{Definición de campo eléctrico}
    \subsection*{Definición de campo eléctrico}
      El campo eléctrico es una propiedad del espacio que rodea una carga eléctrica $q$ y puede ejercer una 
      fuerza eléctrica sobre otra carga de prueba $q_0$ cercana. Se describe mediante vectores que indican la 
      magnitud y dirección de la fuerza eléctrica experimentada por una carga de prueba.
      \begin{equation*}
        \overrightarrow{E}=\frac{\overrightarrow{F}_e}{q_0}\left[\frac{N}{C}\right]
      \end{equation*}

    \addcontentsline{toc}{subsection}{Campo eléctrico de una carga puntual $q$}
    \subsection*{Campo eléctrico de una carga puntual $q$}
      \begin{equation*}
        E=k\frac{q}{r^2}\hat{r}\left[\frac{N}{C}\right]
      \end{equation*}

    \addcontentsline{toc}{subsection}{Campo eléctrico de un grupo de cargas}
    \subsection*{Campo eléctrico de un grupo de cargas}
      \begin{equation*}
        E=k\sum_{i}\frac{q_i}{r_i^2}\hat{r}_i\left[\frac{N}{C}\right]
      \end{equation*}

    \addcontentsline{toc}{subsection}{Campo eléctrico de una distribución de carga continua}
    \subsection*{Campo eléctrico de una distribución de carga continua}
      \begin{equation*}
        E=k_e\int\frac{dq}{r^2}\hat{r}\left[\frac{N}{C}\right]
      \end{equation*}

    \addcontentsline{toc}{subsection}{Aceleración de una carga en un campo eléctrico}
    \subsection*{Aceleración de una carga en un campo eléctrico}
      \begin{equation*}
        a=\frac{qE}{m}
      \end{equation*}

    \addcontentsline{toc}{subsection}{Fuerza sobre una partícula con carga colocada en un campo eléctrico}
    \subsection*{Fuerza ejercida sobre una partícula con carga $q$ colocada en un campo eléctrico}
      \begin{equation*}
        \overrightarrow{F}_e=q_0\overrightarrow{E}[N]
      \end{equation*}

    \addcontentsline{toc}{subsection}{Fuerza sobre una carga de pruebas según la ley de Coulomb}
    \subsection*{Fuerza ejercida por $q$ sobre una carga de pruebas según la ley de Coulomb}
      \begin{equation*}
        \overrightarrow{F}_e=k_e\frac{qq_0}{r^2}\hat{r}[N]
      \end{equation*}
      Dónde $\hat{r}$ es un vector unitario con dirección de $q$ hacia $q_0$.


  \addcontentsline{toc}{section}{Ley de Gauss}
  \section*{Ley de Gauss}

    \addcontentsline{toc}{subsection}{Definición de flujo eléctrico}
    \subsection*{Definición de flujo eléctrico}
      El flujo eléctrico es una medida de la cantidad de líneas de campo eléctrico que atraviesan 
      una superficie dada. Se puede visualizar como la cantidad de líneas que salen o entran en una superficie
      \begin{equation*}
        \Phi_E=\oint_s\overrightarrow{E}\cdot d\overrightarrow{A}=\left[\frac{NM^2}{C}\right]
      \end{equation*}

    \addcontentsline{toc}{subsection}{Flujo eléctrico para superficie sencilla}
    \subsection*{Flujo eléctrico para superficie sencilla}
      \begin{equation*}
        \Phi_E=\overrightarrow{E}\cdot\overrightarrow{A}\left[\frac{NM^2}{C}\right]=EA\cos{\theta}\left[\frac{NM^2}{C}\right]
      \end{equation*}
    Siendo $\overrightarrow{S}$ el vector perpendicular a la superficie y $\theta$ el ángulo conformado por los vectores
    $\overrightarrow{E}$ y $\overrightarrow{A}$.

    \addcontentsline{toc}{subsection}{Ley de Gauss}
    \subsection*{Ley de Gauss}
      \begin{align*}
        \Phi_E &=\frac{\sum q_{int}}{\epsilon_0}\left[\frac{NM^2}{C}\right]\\
        \oint_s\overrightarrow{E}\cdot d\overrightarrow{A}&=\frac{\sum q_{int}}{\epsilon_0}\left[\frac{NM^2}{C}\right]\\
        \Phi_E &=\oint_s\overrightarrow{E}\cdot d\overrightarrow{A}\left[\frac{NM^2}{C}\right]
      \end{align*}

  \addcontentsline{toc}{section}{Potencial eléctrico}
  \section*{Potencial eléctrico}
    
    \addcontentsline{toc}{subsection}{Definición de potencial eléctrico}
    \subsection*{Definición de potencial eléctrico}
      Trabajo que debe realizar un campo eléctrico para mover una carga positiva desde dicho punto hasta el punto de 
      referencia, dividido por unidad de carga de prueba.
      \begin{equation*}
        dw=\overrightarrow{F}\cdot\overrightarrow{ds}=q_0\overrightarrow{E}\cdot\overrightarrow{ds}[J]
      \end{equation*}
\end{document}
